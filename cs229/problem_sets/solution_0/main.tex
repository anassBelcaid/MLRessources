\documentclass[12pt]{article} \usepackage[utf8]{inputenc}
\usepackage{fullpage}
\usepackage{parskip}
\usepackage{amsmath,amssymb,mathtools}
\usepackage{microtype}
\usepackage{booktabs}
\usepackage{graphicx,subcaption}
\usepackage{algorithm}
\usepackage[round]{natbib}
\usepackage{tikz-cd}


\usepackage[most]{tcolorbox}
\newtcolorbox[]{solution}[1][]{%
    breakable,
    enhanced,
    colback=white,
    title=Solution,
    #1
}
\usepackage{xcolor} %hilight
\newcommand{\hilight}[1]{\colorbox{yellow}{#1}}
%% ---------------------------------
%\begin{solution}
%\hilight{TODO}
%\end{solution}
%% ---------------------------------



\usepackage{tikz}
\usetikzlibrary{bayesnet}

\newcommand*\circled[1]{
\tikz[baseline=(char.base)]{\node[shape=circle,draw,inner sep=1pt] (char) {#1};}
}
            
 
% xun
\newcommand{\Gcal}{\mathcal{G}}
\newcommand{\Ical}{\mathcal{I}}
\newcommand{\Hcal}{\mathcal{H}}
\newcommand{\Ucal}{\mathcal{U}}
\newcommand{\Tcal}{\mathcal{T}}
\newcommand{\Ncal}{\mathcal{N}}
\newcommand{\Xcal}{\mathcal{X}}
\newcommand{\Cbs}{\boldsymbol{C}}
\newcommand{\Sbs}{\boldsymbol{S}}
\newcommand{\Pa}{\text{Pa}}
\newcommand{\De}{\text{De}}
\newcommand{\Nd}{\text{Nd}}
            

\title{CS229 Machine Learning (summer 2020): Problem Set 0
{\color{red} v1.1}
}
\author{
\begin{tabular}{rl}
Andrew ID: & Moroccan Student\\
Name: & [Belcaid Anass] \\
Collaborators: & [Working alone]
\end{tabular}
}
\date{\today}


\begin{document}

\maketitle



\section{Gradient and Hessians [0 points]}
\begin{enumerate}
  \item Let $f(x) = \frac{1}{2} x^TAx + b^Tx$, where $A$ is a
    symmetric matrix and $b\in\mathbb{R}^n$ is a vector. What is $\nabla
    f(x)$?
    \begin{solution}
      \begin{equation}
      \nabla f(x) = A x + b 
    \end{equation}
    \end{solution}
  \item Let $f(x) = g(h(x))$, where $g: \mathbb{R}\rightarrow \mathbb{R}$ is
    differentiable and $h\;:\: \mathbb{R}^n\rightarrow \mathbb{R}$ is
    differentiable. What is $\nabla f(x)$?
    \begin{solution}
      \begin{equation}
      \nabla f(x) = \dfrac{\partial g(h(x))}{\partial h(x)} \dfrac{\partial h(x)}{\partial x} 
      = g'(x)\nabla h(x)
    \end{equation}
    \end{solution}
  \item 
  \item Let $f(x) = \frac{1}{2} x^TAx + b^Tx$, where $A$ is a
    symmetric matrix and $b\in\mathbb{R}^n$ is a vector. What is $\nabla^2f(x)$?
    \begin{solution}
      \begin{equation}
      \nabla^2(f(x)) = A
    \end{equation}
    \end{solution}
  \item Let $f(x) = g(a^Tx)$, where $g:\mathbb{R}\rightarrow \mathbb{R}$ is
    continuously differentiable and $a\in \mathbb{R}^n$ is a vector. What  is $\nabla f(x)$ and $\nabla^2(f(x))$?
    \begin{solution}
      \begin{equation}
      \nabla g(x)  = \dfrac{\partial g(x)}{\partial a^Tx} \dfrac{\partial a^T x
      }{\partial x} = g^{'}(a^Tx) a
    \end{equation}

    \begin{equation}
      \nabla^2(g(x))_{(i,j)} = g^{''}(a^Tx) a_i a_j
    \end{equation}
    \end{solution}
\end{enumerate}

\section{Positive definite matrices [0 Points]}%

\begin{itemize}
  \item Let $z\in\mathbb{R}^n$. Show that $A=zz^T$ is positive semidefinite.
    \begin{solution}
      Let $x \in \mathbb{R}^n$,
      \begin{eqnarray*}
        x^T A x & = & x^Tzz^T x \\
                & = & (z^Tx)^T (z^Tx)\\
                & = & \Vert z^Tx \Vert \geq 0
      \end{eqnarray*}
    Which prove that $A$ is \textbf{SPD}.
    \end{solution}

  \item Let $A=zz^T$, what is the null space of $A$? What is the rank of $A$?
    \begin{solution}
      The null space $\Ncal(A)$ is given by:

      \begin{eqnarray}
        \Ncal(A)&=&\left\{ x\in \mathbb{R}^n \;|\; (zz^T)x = 0 \right\}\\
                &=& \left\{ x\in \mathbb{R}^n\;|\; \sum_{i=1}^n x_i
                z_i=0\right\}\\
                &=& z^{\perp}\\
      \end{eqnarray}

      The rank of $A$ is  $n$ - the number of \textbf{non null} elements in $z$.
    \end{solution}

  \item Let $A\in\mathbb{R}^{n,n}$ be a semidefinite matrix and
    $B\in\mathbb{R}^{m,n}$ be an arbitrary matrix. Is $BAB^T$ \textbf{PSD}?
    \begin{solution}
      For any $x\in \mathbb{R}^{m}$, let compute the:

      \begin{equation}
        x^TBAB^Tx  &= & (B^Tx)^T A (B^Tx) \geq 0\\
      \end{equation}
      Since $A$ is \textbf{PSD}, 

    \end{solution}
\end{itemize}

\section{Eigen vectors and Eigen values}%

\begin{itemize}
  \item Suppose $A\in\mathbb{R}^{n,n}$ is diagonalizable, that is, $A=TDT^{-1}$
    for an invertible matrix $T$. Show that $At^{(i)}= \lambda_i t^{(i)}$.

    \begin{solution}
    \begin{eqnarray}
      At^{(i)}  &=& TDT^{-1} t^{(i)}\\
                &=& TD (T^T t^{(i)})\\ 
                &=& TD \begin{pmatrix} \ldots & 1 & \ldots \end{pmatrix}^T\\
                &= & T \begin{pmatrix} \ldots & \lambda_i & \ldots
                \end{pmatrix}^T\\
                & = & \lambda_i t^{(i)}
    \end{eqnarray}  
    \end{solution}
  \item Two simple equations on digitalization.
\end{itemize}

\section{Probability and multivariate Gaussian}%
\label{sec:probability_and_multivariate_gaussian}

Let $X=(X_1,\ldots, X_n)$  is sampled from a \emph{multivariate Gaussian}
distribution with mean $\mu$ and covariance $\Sigma\in S^n_+$.

\begin{itemize}
  \item Describe the random variable $Y=X_1 + X_2 + \ldots + X_n$. What is the
    mean and variance?
\end{itemize}

\bibliography{pgm}
\bibliographystyle{abbrvnat}


\end{document}
